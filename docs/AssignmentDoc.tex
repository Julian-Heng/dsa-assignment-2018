\documentclass[a4paper, 12pt, titlepage]{article}

\usepackage[
    a4paper,
    lmargin=25.4mm,
    rmargin=25.4mm,
    tmargin=20mm,
    bmargin=20mm
]{geometry}

\usepackage{color}
\usepackage{enumitem}
\usepackage{fancyhdr}
\usepackage{graphicx}
\usepackage{inconsolata}
\usepackage{listings}
\usepackage{listing}
\usepackage{nameref}
\usepackage{parskip}
\usepackage{tocloft}

\graphicspath{ {./} }
\newcommand{\code}[1]{\small\texttt{#1}\normalsize}

\definecolor{codegray}{gray}{0.9}
\definecolor{dkgreen}{rgb}{0,0.6,0}
\definecolor{gray}{rgb}{0.5,0.5,0.5}
\definecolor{mauve}{rgb}{0.58,0,0.82}

\lstset
{
    % -- Settings -- %
    breakatwhitespace=true,
    breaklines=true,
    columns=fixed,
    showstringspaces=false,
    xleftmargin=0.65cm,
    % -- Looks -- %
    basicstyle=\footnotesize\ttfamily,
    commentstyle=\color{dkgreen},
    keywordstyle=\color{blue},
    numberstyle=\ttfamily\color{gray},
    stringstyle=\color{mauve}
}

\fancyhf{}
\setlength\columnseprule{0.4pt}
\setlength{\parindent}{0pt}

\title{\huge \textbf{DSA Assignment\\Election Manager Program}}
\author{Julian Heng (19473701)}
\date{October 29, 2018}

\begin{document}

\maketitle
\tableofcontents
\newpage

\pagestyle{fancy}

\fancyhf[HL]{\footnotesize{DSA Assignment - Election Program}}
\fancyhf[FC]{\thepage}
\fancyhf[FL]{\footnotesize{Julian Heng (19473701)}}

\section{UML Diagram}
\fancyhf[HR]{\footnotesize{UML Diagram}}

\includegraphics[
    angle=90,
    width=\textwidth,
    height=0.975\textheight,
    keepaspectratio
]{UML.png}

\newpage

\section{Order of Operations}
\fancyhf[HR]{\footnotesize{Order of Operations}}

The purpose of this section is to describe how the Election Manager is
initialised as so that it can give an idea on how the entire program works.

\subsection{Usage}
The program is expected to be run as follows:

\code{java Driver -d [distance\_file.csv]
-c [candidates\_file.csv] [preference\_file.csv]}

It expects no less than 5 arguments.

\subsection{Initialisation}
Distance and candidates file are to be specifically set, however the
preference file can be inputted without being specified.

After parsing the command line arguments and separate the filenames
into their respective list, the program will start parsing them, starting
with locations, then candidates, and finally the preferences.

The functions called in order are:
\begin{enumerate}
    \item \code{processDistances()}
    \item \code{processNominees()}
    \item \code{processPreference()}
\end{enumerate}

After all files have been processed, they will be contained in linked list,
ready to be passed to the menu options for manipulation by the user.

\subsection{Menu}

\begin{lstlisting}
Please select an option:
    1. List Nominees
    2. Nominee Search
    3. List by Margin
    4. Itinerary by Margin
    5. Exit
\end{lstlisting}

\subsection{Timing}

Each function in the program measures in milliseconds how long it takes
for it to execute. It starts timing after the last user integration and
stops before any printing function occurs. In the initialisation stage,
each file to be parsed will be timed.

\newpage

\section{Classes and Objects}
\fancyhf[HR]{\footnotesize{Classes and Objects}}

The purpose of this section is to document the different classes that are used
in the Election Manager program. Classes with the filename starting with
\code{DSA} are abstract data types while other classes are used for storing
relevant information for the main program.

\subsection{Summary}

List of Objects used:
\begin{itemize}[label={--}]
    \item Menu
    \item HousePreference
    \item Location
    \item Nominee
    \item Trip
    \item VoteStats
\end{itemize}

List of Classes used:
\begin{itemize}[label={--}]
    \item Commons
    \item Driver
    \item ElectionManagerInit
    \item ElectionManager
    \item FileIO
    \item Input
\end{itemize}

List of Abstract Data Types used:
\begin{itemize}[label={--}]
    \item DSAGraph
    \item DSAHashTable
    \item DSALinkedList
    \item DSAMaxHeap
    \item DSAMinHeap
    \item DSAMergeSort
\end{itemize}
\newpage

\subsection{Objects}

These are classes that create an object. In other words, they contain a
constructor and holds values.

\subsubsection{Menu}

A menu object used to display menu. Created with a given size, then
options are given to it. Uses only \textit{arrays} as we do not expect to add
any new options to the menu as they are already predetermined.

\subsubsection{HousePreference}

HousePreference class is used to store relevant information from the
preference file. A HousePreference object is identified by the name
of the division. In addition to storing the division name and ID,
a linked list is also used to store nominees involved in those divisions
as well as the total number of votes, including informal votes. In the list
of nominees, we do not copy them from the list of nominees in the main program
as we will be modifying them as we are parsing the preference file.

A linked list is used as the number of nominees is not fixed, thus a linked
list is used for conveniently adding more nominees that are part of a division.
In addition, we are not expecting to access a particular nominee in the list,
only iterating through them.

\subsubsection{Location}

In it's simplest form, just a object that records details about a location.
This includes storing state, division, latitude and longitude. Mainly used
as the value stored in the vertex of a graph. It's division is the identifier
of the location class, thus the division is used as the label of the vertex.
Used in conjunction with the Trip class in the graph. Does not contain any
abstract data types.

\subsubsection{Nominee}

Contains information about a nominee or candidate. Other than storing the
state, division, party and name, it also keeps count of the amount of votes
for that one nominee. Information, that is not given in the candidates file,
but in the preference file. In relation to the main program, multiple nominees
are expected to be made when using the program, thus the nominee class is
expected to be stored in a linked list in the main program. Does not contain
any abstract data types.

\subsubsection{Trip}

Just like the Location class, it stores information involving the means of
travelling from one Location class to another Location class. It stores
the type of transport, the distance in meters and the duration in seconds.
Used in conjunction with the Location class in the graph, as the value
stored in the edges of the graph. It's label would be the two locations
connected to the edge and the edge weight would be the duration of the trip.
Does not contain any abstract data types.

\subsubsection{VoteStats}

In order to simplify calculating the margin and creating an itinerary, a
VoteStats class is used to store the voting results of each division for
a specific party. Mutator methods involves calculating the percent of votes
for one party as well as updating the margin. Does not contain any abstract
data types.

\subsubsection{Miscellaneous Objects Design}

All objects, excluding the Menu class, have setters and getters. Some of them
have a \code{toString()} method, only if we are expecting to save the contents
of the object to a text file.

Because we are parsing information from a text file, we will also be
constructing these objects with string inputs, as to not complicate
the main program. Thus converting from String to other data types are passed
to the classes to handle.

\newpage

\subsection{Classes}

These are classes that do not create an object, in other words, they provide
functions.

\subsubsection{Commons}

The purpose of the Commons class is to provide common or generic utilities
and functions for use to the other classes. Hence the name Commons.

\paragraph{convertTimeToString} \hspace{0pt}

Converts from seconds to a more reasonably distinguishable time format.
Does not use any abstract data types.

\paragraph{printCsvTable} \hspace{0pt}

Converts a csv file stored in an \textit{array} to a neatly formatted table.
Takes in 2 \textit{arrays} as imports, one for the contents and the other for
the headers. We use an \textit{array} instead of a linked list because we
expect the csv \textit{array} to be stored as a file, which takes in an
\textit{array}. Hence, this function will accept them as an \textit{array}
and does not use any abstract data type.

\paragraph{calcMaxStrignArrLength} \hspace{0pt}

Finds the longest string in an \textit{array} of strings. Only uses
\textit{arrays} as \code{printCsvTable()} uses \textit{arrays} . Does not use
any abstract data type.

\paragraph{saveCsvToFile} \hspace{0pt}

Saves a csv file stored in an \textit{array} to a text file. Takes in 2
imports, the contents of the file and the default filename if a filename
is not provided. Does not use any abstract data type.

\paragraph{header} \hspace{0pt}

Print out a header for a given message. Does not use any abstract data type.

\paragraph{formatPadding} \hspace{0pt}

A wrapper function to simply return back a whitespaces of a size dictated by
the padding variable. An alternative function takes in an additional
import to change the whitespace to a different character. Does not use any
abstract data type.

\paragraph{generateLine} \hspace{0pt}

Very similar to \code{formatPadding()} as it uses similar code, but instead
takes in a size instead and returns just a line. Does not use any abstract
data type.

\paragraph{compareIntString} \hspace{0pt}

A wrapper function to compare a string representation of an integer with
an integer. This is used because when parsing a file, it would be ``simpler''
to just call this function instead of parsing the string to an integer.
Does not contain any abstract data type.

\paragraph{toMilliseconds} \hspace{0pt}

Convert nanoseconds to milliseconds to 3 decimal places. Does not contain
any abstract data type.

\newpage

\subsubsection{Driver}

The purpose of the Driver class is to ``drive'' the Election Manager program,
handling the initialisation and the user actions of the program.

\paragraph{main} \hspace{0pt}

Check if number of arguments passed are valid, then parse input files and
run menu.

\paragraph{parseArgs} \hspace{0pt}

Sort out which files belong to which objects. Because there are command line
switches for specifying which file are distances, nominees or preferences,
we need some way to process them individually. We used a linked list to
store the filename because we do not need to access a particular filename
and we will need to process through all of the items in the list anyways.
Thus, a linked list is better fit for this task.

\paragraph{prepareLists} \hspace{0pt}

For each filename in the distance, nominee and house preference file, we
will need to process and parse those files into 3 objects; a \textit{graph}
representing the map, a \textit{linked list} for the nominees and a
\textit{linked list} for the house preferences. We will provide rationale
for using \textit{linked list} when they are being used.

\paragraph{menu} \hspace{0pt}

Provide a user interface for the user to interact with. See above for
an example of the menu.

\newpage

\subsubsection{ElectionManagerInit}

The purpose of the ElectionManagerInit is to both de-clutter the Driver
class and to initialise objects for the main program to use.

\paragraph{processDistance} \hspace{0pt}

For a given filename, read and process each entry in the file
and create vertices and edges to be added to a \textit{graph} to make a map.
Every entry in a distance file is expected to be able to create 2
Location class and create an edge in between them. We can freely add
an already existing vertex in the \textit{graph} because the \textit{graph}
will handle duplicate entries in that it will ignore it. We use a
\textit{graph} because there are no other abstract data types that are
appropriate for this task.

We also assume that we can travel from one location to another location
and back to the original location, making this \textit{graph} an undirected
\textit{graph}.

\paragraph{processNominee} \hspace{0pt}

For a given filename, read and process each entry in the file and create
nominee objects to be added to a \textit{linked list}. We can pass the
entire csv line to the constructor of the Nominee class because, unlike
the house preference class, each line can only represent one single nominee
class as they do not rely on information from other lines in the nominees
file.

\paragraph{processPreference \\
           sortByDivisionId \\
           getNomineeFromState \\
           getNomineeFromDivisionId \\
           getNomineeFromCandidateId \\
           getPreferenceFromDivisionId
} \hspace{0pt}

Out of all of the preprocessing and initialisation, the house preference
is perhaps one of the most involved.

In order to successfully make a house preference object, we need to:
\begin{enumerate}
    \item Read the file to a \textit{linked list}
    \item Get unique values for the division
    \item Get the vote statistics for each division
    \item Set the nominees list within the preference object by referencing
          the nominees list created before this
    \item Add the created preference object to the preference
          \textit{linked list} and repeat until file is completely processed
\end{enumerate}

If for any reason \code{processNominee()} failed, then creating a valid
house preference \textit{linked list} will most likely fail as well.

For getting unique values for the division ID, we use a \textit{hash table}.
\textit{Hash tables} are perfect for getting unique values as hash tables
store the key as a hash, and will not allow inserting identical keys into
the \textit{hash table}.

In order to increase the speed of making the house preference object, we
have applied a sort to the input file according to division ID so that the
relevant ID for each preference object are grouped together. This way, we can
skip entries that do not contain the division ID and stop reading the file
after the division ID does not match.

The sort used to make this possible is the \textit{heap sort}. We used a heap
sort because we don't require to convert the current \textit{linked list}
structure to an \textit{array} as a \textit{heap} already inserts new objects
into an \textit{array}, thus only needing to convert back to a linked list
after the sort, as well as the only sort available that we have that could
sort objects according to a key. We do not have to worry about
\textit{heap sort} being unstable because the order does not matter, all the
files will get processed all the same. The same rationale is used later when
listing nominees.

\newpage

\subsubsection{ElectionManager}

The ElectionManager Class is the core of the ElectionManager program. This
class contains methods that will search and process information from the
objects created in \code{ElectionManagerInit}.

\paragraph{listNominees \\
           processFilterOptions \\
           getFilter \\
           searchNominee1 \\
           generateSortLine
} \hspace{0pt}

When the user selects this option (1), the program will prompt the user
to apply a filter or sort the nominees in a choice of 4 attributes. If the
user did not input anything, then it will just display all nominees in
whatever order they were arranged before.

For the filter, we have \code{getFilter()} to return a regular expression that
will match everything if user did not input anything. Otherwise, it would have
return what the user have entered.

For the order, we used a \textit{heap} to sort our nominee classes. The
rationale for using \textit{heap sort} is that it can sort objects as well
as according to a key. In terms of sorting by the order the user inputed, this
can be solved by generating a sort line. By calling \code{generateSortLine()},
it would return a string that contains the attribute of the nominee in the
order the user selected. This string would then become the key when inserting
to the heap, thus allowing \textit{heap sort} to work.

The reason we put the nominee classes to a linked list is that we have
to process and iterate through the entire list anyway. Thus defeating the
advantage of the \textit{array} of accessing any element instantly. In
addition to the linked list having no ``size'' limit, a linked list is more
appropriate for storing nominee classes. However, it is very important to note
that when saving the file or printing the csv table, it is a string
\textit{array}. This is due to how we create the itinerary which will be
explained in that relevant section.

Because we have regular expressions that are capable of matching everything
if the user did not input anything, we can simply use the same variable as the
search term, passing it into the \code{matches()} method in the string class.
\code{searchNominee1()} specifically searches the state, party name or
abbreviation, and the division name or ID.

\pagebreak
\paragraph{searchNominees \\
           ProcessFilterOptions \\
           getFitler \\
           searchNominee2
} \hspace{0pt}

When the user selects this option (2), the program will prompt the user
to apply a filter and the search term, it being the first few characters
of the nominee's surname. This is essentially a subset of what
\code{listNominees()} accomplishes. As such, the same methods used in
\code{listNominees()} are also applied here. \code{searchNominee2()}
specifically searches the start of the nominee's surname, the state, and the
party name or abbreviation.

\paragraph{listPartyMargin \\
           getPartyFilter \\
           getMarginLimit \\
           \_listPartyMargin \\
           calcVotes
} \hspace{0pt}

When the user selects this option (3), the program will prompt the user
for a party name and the margin limit. After that has been entered, it will
list divisions of which that particular party have a margin that is within
plus or minus the margin limit.

If the user did not enter any party name, then it will exit out to the menu
immediately. After both party filter and margin limit are entered, it will
call \code{calcVotes()} to get the voting statistics for one particular party
in the form of a list. Then, it will iterate through the list and append
any divisions to a list whose margin is less than the user set limit. If
the list is empty, it indicates that there are no results for the given
margin limit, thus prompting the user that the filters and/or limit is
invalid. Otherwise, print and save table to file.

We used a linked list when getting search result in the same rationale as
listing nominees in that we do not know how many matches there are. So instead
of counting matches and creating an \textit{array} of that size, then adding
it to the \textit{array}, causing the algorithm to iterate through the list
of voting statistics twice, it would be much more easier to add to the end
of a list instead.

\paragraph{createItinerary \\
           getPartyFilter \\
           getMarginLimit \\
           \_listPartyMargin \\
           calcVotes \\
           printItinerary
} \hspace{0pt}

When the user selects this option (4), the program will prompt the user
for a party name and the margin limit. After that has been entered, it will
list divisions of which that particular party have a margin that is within
plus or minus the margin limit. After this, the user will be prompted to enter
the index of the locations they would want to visit and a path to visit these
locations will be made.

Since creating an itinerary contains a super set of functions contained in
\code{listPartyMargin()}, we would reuse some functions, particularly
\code{calcVotes()} and \code{\_listPartyMargin()}.

While most of the abstract data types we have used so far are
\code{linked list}, when selecting locations from a list, it is best suited
to use an \textit{array} for instant access to specific elements. Hence why
when saving to a file we always use an \textit{array}.

When optimising the path, we did it in the most simplest way possible. We
would sort by the index of the appearance of the locations. This depends on
the output of \code{\_listPartyMargin()} to already be sorted by state so that
all of the locations are at the very least grouped together by state. The sort
used is a \textit{merge sort} as it has the time complexity of $O(n\log{}n)$
across all cases. In terms of memory usage, it would be much more higher due
to making another \textit{array} of the same size, but given the context of
the entire program, it would seem that duplicating an \textit{array} of the
same size as the number of locations (at worst) would seem trivial.

When creating a path, we made the executive decision to use Dijkstra's
algorithm to visit every locations selected. However, it's a very crude and
simple implementation in that it would only connect to the next selected
location. If given more time, we could have performed Dijkstra on each of
the selected location against each other, then select the shortest route
after iterating through all the possible paths. However, we're stuck with
this instead.

\subsubsection{FileIO}

The FileIO class is to provide read and write methods to other classes.
When reading from a file, it stores it to a \textit{linked list}, but
when writing to a file, it uses an \textit{array}. This is due to how
we have structured the main program to iterate through all items in a file,
but require the output of a file to be an \textit{array} for
\code{createItinerary()}. Hence, all methods that call \code{writeText()} will
require the output file to be in an \textit{array}.

\subsubsection{Input}

The Input class is to provide inputting methods for certain data type to other
classes. Does not use any abstract data types.

\newpage

\subsection{Abstract Data Types}

These are classes that are abstract data types that are used within the
program.

\subsubsection{DSAGraph}

Within the \textit{graph}, there are 2 private classes, vertex and an edge.
For the adjacent and edge list, we have decided to use a \textit{linked list}.
Ideally, it would be better to use a \textit{hash table} because we can't
insert identical labelled vertexes or edges into the \textit{graph}. However,
due to time constraints, we have decided to stick with a \textit{linked list}.

In the Dijkstra's algorithm implementation, we have added 2 more class fields
for the vertex:

\begin{itemize}[label={--}]
    \item \code{distanceFromSource}
    \item \code{prevVertex}
\end{itemize}

\code{distanceFromSource} stores an integer that represents the total distance
it would take to go from a given source vertex to the vertex this variable is
stored in. \code{prevVertex} represents the vertex needed to be accessed by in
order to get the shortest path to this vertex.

\subsubsection{DSAHashTable}

Within the \textit{hash table}, there is only 1 private class, the
\textit{hash entry} class which contains only the key, the object value stored,
and the state of the entry. A \textit{hash table} utilises an array because
we need instant access to any element in the table depending on the key.
The hash algorithm used is the Shift-Add-XOR hash. The step hash algorithm
uses 5 as the maximum step.

We used \textit{hash tables} whenever we need to get unique values out of a
linked list or array.

\subsubsection{DSALinkedList}

Within the \textit{linked list}, there are 2 private class, the
\textit{list node} and the \textit{iterator}. The \textit{linked list} is the
most used abstract data structure in the main program. The decision as to
why it is used over arrays is for a number of reasons.

\begin{enumerate}
    \item
    It does not require a known size before it can be used.

    We can remedy this with \textit{arrays}, however it would require to
    iterate through the \textit{array} twice. And even then, we can't
    freely add new items when the \textit{array} becomes full. Again, we
    can remedy this, but it would require creating a new \textit{array}
    and moving the elements from one \textit{array} to the next.

    \item
    Stored items does not require instant access

    We normally used \textit{linked list} to store items that does not
    need instant access. In cases where we need to process an entire set
    of data, each element will be visited anyway, thus a \textit{linked list}
    would be better suited for this task instead.\\

    \item
    Adding items are $O(n)$

    With adding new items to the end of an \textit{array}, you would need to
    keep count of the last used item. When adding to the beginning of the
    \textit{array}, then \textit{all} the elements will need to be shuffled.
    \textit{Linked list} does not have any of these problems as we can
    add a new \textit{list node} to either the beginning or the end of the
    \textit{linked list}.
\end{enumerate}

\subsubsection{DSAMaxHeap \& DSAMinHeap}

This program features 2 different variations of the heap because both
are used for different purpose. The \textit{min} and \textit{max} variants
refers to the type of priority queues, with the \textit{min} being a minimum
\textit{priority queue} and the \textit{max} being a maximum
\textit{priority queue}. The \textit{max} variant is used primarily for the
heap sort and the \textit{min} variant is used only in the Dijkstra algorithm
in the \textit{graph}.

The rationale for using \textit{heap sort} is described above. In summary it
allows sorting of objects easily, as well as having the time complexity of
$O(n\log{}n)$. Stability of the sort does not matter in this case as the order
of objects with identical keys does not matter.

For Dijkstra's algorithm, we used a minimum \textit{priority queue} as we
require dequeuing the smallest distance, or key, from the queue.

\subsubsection{DSAMergeSort}

\textit{Merge sort} is used to sort user inputs for inputting locations to
visit in \code{createItinerary}. We decided to use merge sort because, like
\textit{heap sort}, has the time complexity of $O(n\log{}n)$. Since we're only
sorting integers, we do not need \textit{heap sort} for this task.

\end{document}
